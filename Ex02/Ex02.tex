\documentclass[a4paper,16pt]{jsarticle}

% 余白の設定
\setlength{\textwidth}{\fullwidth}
\setlength{\textheight}{40\baselineskip}
\addtolength{\textheight}{\topskip}
\setlength{\voffset}{-0.2in}
\setlength{\topmargin}{0pt}
\setlength{\headheight}{0pt}
\setlength{\headsep}{0pt}

% パッケージ
\usepackage[dvipdfmx]{hyperref,graphicx}		% 画像
\hypersetup{
	colorlinks=true, % リンクに色をつけない設定
	bookmarks=true, % 以下ブックマークに関する設定
	bookmarksnumbered=true,
	pdfborder={0 0 0},
	bookmarkstype=toc
}
\usepackage{amsmath, amssymb}		% ギリシャ文字
\usepackage{bm}				% 数式 \bm{a} aベクトル
\usepackage{comment}			% コメント
\usepackage{siunitx}			% SI単位
\usepackage{framed}			% 枠組み
\usepackage{braket}     % ブラケット記法

\newtheorem{theorem}{定理}
\newtheorem{proof}{証明}
\newcommand{\qed}{\qquad $\blacksquare$}

\begin{comment}
 複数行に渡るコメントの書き方
 \usepackage{comment}が必要。
\end{comment}

% section前に改ページ
\makeatletter
\def\section{\newpage\@startsection {section}{1}{\z@}{-3.5ex plus -1ex minus -.2ex}{2.3 ex plus .2ex}{\Large\bf}}
\makeatother

% 数式番号をいい感じに
\makeatletter
% \renewcommand{\theequation}{\arabic{chapter}-\arabic{section}-\arabic{equation}}
	\renewcommand{\theequation}{\arabic{section}-\arabic{equation}}
  \@addtoreset{equation}{section}
\makeatother

\title{Ex02}
\author{ゆきちゃん}
\date{\today}

\begin{document}
\maketitle

\begin{itemize}
	\item The input step signal $x(t)$ and the impulse response $h(t)$ of a continuous time LT$1$ system are given by
	\begin{equation}
		x(t) = u(t),~~~~h(t) = e^{-\alpha t}u(t),~~~~\alpha > 0
	\end{equation}
\end{itemize}

\begin{enumerate}
	\item Compute the output $y(t) = x(t)*h(t) = \int_{-\infty}^\infty x(\tau)h(t-\tau) d\tau$
\end{enumerate}

Answer:

\begin{align}
	y(t) &= \int_{-\infty}^\infty x(\tau)h(t-\tau) d\tau \\
	&= \int_{-\infty}^\infty u(\tau)e^{-\alpha (t-\tau)}u(t-\tau) d\tau \\
	&= \int_{-\infty}^\infty e^{-\alpha (t-\tau)}u(\tau)u(t-\tau) d\tau
\end{align}

$t < 0$の時、
$u(\tau)u(t-\tau) = 0$なので、$y(t) = 0$

$0 \leq t$の時、
\begin{equation*}
	u(\tau)u(t-\tau) =
	\begin{cases}
		1 & (0 \leq \tau \leq t) \\
		0 & otherwise
	\end{cases}
\end{equation*}
よって、
\begin{align}
	y(t) &= \int_0^t e^{-\alpha (t-\tau)} d\tau \\
	&= \dfrac{1}{\alpha}[ 1 - e^{-\alpha t}]
\end{align}

したがって、
\begin{equation}
	y(t) = \dfrac{1}{\alpha}[ 1 - e^{-\alpha t}]u(t)
\end{equation}

\end{document}

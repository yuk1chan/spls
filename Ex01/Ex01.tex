\documentclass[a4paper,16pt]{jsarticle}

% 余白の設定
\setlength{\textwidth}{\fullwidth}
\setlength{\textheight}{40\baselineskip}
\addtolength{\textheight}{\topskip}
\setlength{\voffset}{-0.2in}
\setlength{\topmargin}{0pt}
\setlength{\headheight}{0pt}
\setlength{\headsep}{0pt}

% パッケージ
\usepackage[dvipdfmx]{hyperref,graphicx}		% 画像
\hypersetup{
	colorlinks=true, % リンクに色をつけない設定
	bookmarks=true, % 以下ブックマークに関する設定
	bookmarksnumbered=true,
	pdfborder={0 0 0},
	bookmarkstype=toc
}
\usepackage{amsmath, amssymb}		% ギリシャ文字
\usepackage{bm}				% 数式 \bm{a} aベクトル
\usepackage{comment}			% コメント
\usepackage{siunitx}			% SI単位
\usepackage{framed}			% 枠組み
\usepackage{braket}     % ブラケット記法

\newtheorem{theorem}{定理}
\newtheorem{proof}{証明}
\newcommand{\qed}{\qquad $\blacksquare$}

\begin{comment}
 複数行に渡るコメントの書き方
 \usepackage{comment}が必要。
\end{comment}

% section前に改ページ
\makeatletter
\def\section{\newpage\@startsection {section}{1}{\z@}{-3.5ex plus -1ex minus -.2ex}{2.3 ex plus .2ex}{\Large\bf}}
\makeatother

% 数式番号をいい感じに
\makeatletter
% \renewcommand{\theequation}{\arabic{chapter}-\arabic{section}-\arabic{equation}}
	\renewcommand{\theequation}{\arabic{section}-\arabic{equation}}
  \@addtoreset{equation}{section}
\makeatother

\title{Ex01}
\author{ゆきちゃん}
\date{\today}

\begin{document}
\maketitle


\begin{itemize}
	\item Let $x(t)$ be the complex exponential signal $x(\tau) = e^{j\omega_0 t}$ with radian frequency $\omega_0$ and fundamental period $T_0 = \dfrac{2\pi}{\omega_0}$.
	\item Consider the discrete-time signal $x[n]$ obtained by uniform sampling of $x(t)$ with sampling interval $T_s$. That is, $x[n] = x(nT_s) = e^{j\omega_0 n T_s}$.
	\item Find the condition on the value of $T_s$, so that $x[n]$ is preiodic.
\end{itemize}

answer:(maybe)

$x[n] = x[n + mT_0]$となればいいはず?なので、

\begin{align}
	x[n + mT_0]
	&= x((n+mT_0)T_s) \\
	&= e^{j\omega_0(n+mT_0)T_s} \\
	&= e^{j\omega_0nT_s}e^{j\omega_0mT_0T_s} \\
	&= e^{j\omega_0nT_s} \\
	&= x[n]
\end{align}

つまり、
\begin{equation}
	e^{j\omega_0mT_0T_s} = 1 = e^{j2\pi k}
\end{equation}

よって、
\begin{equation}
	\omega_0mT_0T_s = 2\pi k
\end{equation}

したがって
\begin{equation}
	T_s = \dfrac{2\pi k}{\omega_0mT_0}
\end{equation}

ただし、$m,k$は整数

多分...

\end{document}

\documentclass[a4paper,16pt]{jsarticle}

% 余白の設定
\setlength{\textwidth}{\fullwidth}
\setlength{\textheight}{40\baselineskip}
\addtolength{\textheight}{\topskip}
\setlength{\voffset}{-0.2in}
\setlength{\topmargin}{0pt}
\setlength{\headheight}{0pt}
\setlength{\headsep}{0pt}

% パッケージ
\usepackage[dvipdfmx]{hyperref,graphicx}		% 画像
\hypersetup{
	colorlinks=true, % リンクに色をつけない設定
	bookmarks=true, % 以下ブックマークに関する設定
	bookmarksnumbered=true,
	pdfborder={0 0 0},
	bookmarkstype=toc
}
\usepackage{amsmath, amssymb}		% ギリシャ文字
\usepackage{bm}				% 数式 \bm{a} aベクトル
\usepackage{comment}			% コメント
\usepackage{siunitx}			% SI単位
\usepackage{framed}			% 枠組み
\usepackage{braket}     % ブラケット記法

\newtheorem{theorem}{定理}
\newtheorem{proof}{証明}
\newcommand{\qed}{\qquad $\blacksquare$}

\begin{comment}
 複数行に渡るコメントの書き方
 \usepackage{comment}が必要。
\end{comment}

% section前に改ページ
\makeatletter
\def\section{\newpage\@startsection {section}{1}{\z@}{-3.5ex plus -1ex minus -.2ex}{2.3 ex plus .2ex}{\Large\bf}}
\makeatother

% 数式番号をいい感じに
\makeatletter
% \renewcommand{\theequation}{\arabic{chapter}-\arabic{section}-\arabic{equation}}
	\renewcommand{\theequation}{\arabic{section}-\arabic{equation}}
  \@addtoreset{equation}{section}
\makeatother

\title{Ex03}
\author{ゆきちゃん}
\date{\today}

\begin{document}
\maketitle

\begin{itemize}
	\item Given a rectangular pulse sugnal $x(t)$ defined by
	\begin{equation}
		x(t) =
		\begin{cases}
			1, & |t| < \alpha \\
			0, & |t| > \alpha
		\end{cases}
	\end{equation}
\end{itemize}

\begin{enumerate}
	\item Find the Fourier transform $X(\omega)$ of $x(t)$
\end{enumerate}

Answer:
\begin{align}
	X(\omega) &= \mathcal{F}[x(t)] \\
	&= \int_{-\infty}^\infty x(t) e^{-j\omega t}dt \\
	&= \int_{-a}^a e^{-j\omega t}dt
\end{align}

$\omega = 0$の時
\begin{align}
	X(\omega) &= \int_{-a}^a dt \\
	&= 2a
\end{align}

$\omega \neq 0$の時
\begin{align}
	X(\omega) &= \left[\dfrac{1}{-j\omega} e^{-j\omega t}\right]_{-a}^a\\
	&= \dfrac{1}{-j\omega}\left(e^{-j\omega a} - e^{j\omega a}\right) \\
	&= \frac{2}{\omega} \sin(\omega a) \\
	&= 2a \mathrm{sinc}(\omega a)
\end{align}

\end{document}

\documentclass[a4paper,16pt]{jsarticle}

% 余白の設定
\setlength{\textwidth}{\fullwidth}
\setlength{\textheight}{40\baselineskip}
\addtolength{\textheight}{\topskip}
\setlength{\voffset}{-0.2in}
\setlength{\topmargin}{0pt}
\setlength{\headheight}{0pt}
\setlength{\headsep}{0pt}

% パッケージ
\usepackage[dvipdfmx]{hyperref,graphicx}		% 画像
\hypersetup{
	colorlinks=true, % リンクに色をつけない設定
	bookmarks=true, % 以下ブックマークに関する設定
	bookmarksnumbered=true,
	pdfborder={0 0 0},
	bookmarkstype=toc
}
\usepackage{amsmath, amssymb}		% ギリシャ文字
\usepackage{bm}				% 数式 \bm{a} aベクトル
\usepackage{comment}			% コメント
\usepackage{siunitx}			% SI単位
\usepackage{framed}			% 枠組み
\usepackage{braket}     % ブラケット記法

\newtheorem{theorem}{定理}
\newtheorem{proof}{証明}
\newcommand{\qed}{\qquad $\blacksquare$}

\begin{comment}
 複数行に渡るコメントの書き方
 \usepackage{comment}が必要。
\end{comment}

% section前に改ページ
\makeatletter
\def\section{\newpage\@startsection {section}{1}{\z@}{-3.5ex plus -1ex minus -.2ex}{2.3 ex plus .2ex}{\Large\bf}}
\makeatother

% 数式番号をいい感じに
\makeatletter
% \renewcommand{\theequation}{\arabic{chapter}-\arabic{section}-\arabic{equation}}
	\renewcommand{\theequation}{\arabic{section}-\arabic{equation}}
  \@addtoreset{equation}{section}
\makeatother

\title{Ex05}
\author{ゆきちゃん}
\date{\today}

\begin{document}
\maketitle

\begin{itemize}
	\item Given the Fourier transform $X(\omega)$ of $x[n]$
	\begin{equation}
		X(\omega) = \dfrac{1}{(1-ae^{-j\omega})^2},~~~~|a| < 1
	\end{equation}
\end{itemize}
\begin{enumerate}
	\item Find the inverse Fourier transform $x[n]$
\end{enumerate}

Hint:please use the convolution theorem and Fourier transform pair $a^n u[n] \iff \dfrac{1}{1-ae^{-j\omega}}, |a| < 1$

Answer:

\begin{align}
	x[n] &= \mathcal{F}^{-1}\{\mathcal{F}[a^nu[n]]\mathcal{F}[a^nu[n]]\} \\
	&= a^nu[n]*a^nu[n] \\
	&= \sum_{m=-\infty}^\infty a^m u[m]a^{n-m}u[n-m]
\end{align}

ここで、$u[m]u[n-m] = 1$になるための条件は
$m \geq 0$かつ$n-m \geq 0$より、$n \geq m \geq 0$であるので、

\begin{align}
	x[n] &= \sum_{m = 0}^n a^m a^{n-m} \\
	&= a^n\sum_{m = 0}^n 1 \\
	&= (n+1)a^n
\end{align}



\end{document}

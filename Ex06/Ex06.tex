\documentclass[a4paper,16pt]{jsarticle}

% 余白の設定
\setlength{\textwidth}{\fullwidth}
\setlength{\textheight}{40\baselineskip}
\addtolength{\textheight}{\topskip}
\setlength{\voffset}{-0.2in}
\setlength{\topmargin}{0pt}
\setlength{\headheight}{0pt}
\setlength{\headsep}{0pt}

% パッケージ
\usepackage[dvipdfmx]{hyperref,graphicx}		% 画像
\hypersetup{
	colorlinks=true, % リンクに色をつけない設定
	bookmarks=true, % 以下ブックマークに関する設定
	bookmarksnumbered=true,
	pdfborder={0 0 0},
	bookmarkstype=toc
}
\usepackage{amsmath, amssymb}		% ギリシャ文字
\usepackage{bm}				% 数式 \bm{a} aベクトル
\usepackage{comment}			% コメント
\usepackage{siunitx}			% SI単位
\usepackage{framed}			% 枠組み
\usepackage{braket}     % ブラケット記法

\newtheorem{theorem}{定理}
\newtheorem{proof}{証明}
\newcommand{\qed}{\qquad $\blacksquare$}

\begin{comment}
 複数行に渡るコメントの書き方
 \usepackage{comment}が必要。
\end{comment}

% section前に改ページ
\makeatletter
\def\section{\newpage\@startsection {section}{1}{\z@}{-3.5ex plus -1ex minus -.2ex}{2.3 ex plus .2ex}{\Large\bf}}
\makeatother

% 数式番号をいい感じに
\makeatletter
% \renewcommand{\theequation}{\arabic{chapter}-\arabic{section}-\arabic{equation}}
	\renewcommand{\theequation}{\arabic{section}-\arabic{equation}}
  \@addtoreset{equation}{section}
\makeatother

\title{Ex06}
\author{ゆきちゃん}
\date{\today}

\begin{document}
\maketitle

\begin{itemize}
	\item Given an ideal low-pass filter friquency response
	\begin{equation}
		H(\omega) =
		\begin{cases}
			1, & |\omega| < \omega_c \\
			0, & |\omega| > \omega_c
		\end{cases}
	\end{equation}
\end{itemize}

\begin{enumerate}
	\item Find its impulse response $h[n]$
\end{enumerate}

Hint:
\begin{enumerate}
	\item use inverse Fourier transform definition to find $h[n]$
	\item $\mathrm{sinc}$ function is defined i MATLAB as $
	\mathrm{sinc}(t) =
	\begin{cases}
	 \frac{\mathrm{sinc}(\pi t)}{\pi t}, & t \neq 0 \\
		1, & t = 0
	\end{cases}
	$
\end{enumerate}

Answer:

\begin{align}
	h[n] &= \dfrac{1}{2\pi}\int_{\infty}^\infty H(\omega)e^{j\omega n} d\omega \\
	&= \dfrac{1}{2\pi}\int_{-\omega_c}^{\omega_c}e^{j\omega n} d\omega \\
	&= \dfrac{1}{2\pi}\dfrac{1}{jn}(e^{j\omega_c n} - e^{-j\omega_c n}) \\
	&= \dfrac{1}{2\pi}\dfrac{1}{jn} 2j\sin(\omega_c n) \\
	&= \dfrac{1}{\pi n}\sin(\omega_c n) \\
	&= \dfrac{\omega_c}{\pi} \dfrac{\sin(\omega_c n)}{\omega_c n} \\
	&= \dfrac{\omega_c}{\pi} \mathrm{sinc}(\omega_c n)
\end{align}

\end{document}
